
\section{Kennismaken met de Arduino omgeving}

De Arduino omgeving oftewel Arduino IDE (Integrated Development Environment) kan je gebruiken om de Microbit te programmeren. 
%Mocht je de software nog niet geïnstalleerd hebben, volg dan de instructies in “Installatiehandleiding Arduino software voor BBC Microbit”.

We gebruiken de Arduino IDE omdat deze op beginnersniveau zeer veel gebruikt wordt en er veel voorbeeldcode te vinden is.
 
Eigenlijk maakt de Arduino IDE gebruik van de taal C++, dit is een Object Oriented uitbreiding van de taal C. Het C++ / Object Oriented deel van de programmeertaal zie je soms terug in instructies zoals Serial.begin(9600); De punt tussen Serial en begin is hier een indicatie van.

Java en C++ zijn object georiënteerde talen. Daarover leer je meer als je kiest voor de differentiaties “Software Engineering” of “Networks \& System Engineering”. Voor dit practicum houden we het eenvoudig. We gebruiken de taal zoveel mogelijk als klassieke “imperatieve” programmeertaal waarbij gewerkt wordt met een reeks opeenvolgende instructies.



\subsection{Het installeren van de Arduino omgeving}

VOER ONDERSTAANDE INSTALLATIE THUIS UIT – DIT DUURT 1,5 UUR!!!

\begin{enumerate}
	\item Download de Arduino software van \href{https://www.arduino.cc/en/software}{https://www.arduino.cc/en/software}, zoals je kan zien zijn er ook versies voor Linux en de Mac.
	\item  Bevestig alle vragen, klik maar door.
	\item Bij het het opstarten van arduino zal gevraagd worden om de libary's te updaten. Update de libary's.
\end{enumerate}
Na het opstarten verschijnt het scherm zoals in figuur \ref{fig:arduino_s1} wordt weergegeven.

\begin{figure}[h!]
	\captionsetup{justification=centering}
	\includegraphics[width=0.50 \linewidth]{figuren/arduino_s1}
	\centering
	\caption{De Arduino omgeving met een leeg scherm.}
	\label{fig:arduino_s1}
\end{figure}
Waarin de functie \lstinline |void setup() {}| één keer wordt uitgevoerd en de functie \lstinline |void loop() {}| continu wordt uitgevoerd.


Als eerste kunnen een aantal instellingen aangepast worden. Ga naar File $\rightarrow$ Preferences, en pas de instellingen aan zoals te zien is in figuur \ref{fig:arduinoPref}.
\begin{figure}[H]
	\captionsetup{justification=centering}
	\includegraphics[width=0.50 \linewidth]{figuren/arduinoPref}
	\centering
	\caption{preferences scherm.}
	\label{fig:arduinoPref}
\end{figure}

\begin{itemize}
	\item Indien de Arduino omgeving Engelstalig is, kan deze gewijzigd worden naar het Nederlands. Het is handig om tijdens de installatie de Engelstalige omgeving te gebruiken.
\end{itemize}
\subsection{De arduino omgeving geschikt maken voor de microbit.}

De firma \href{https://www.adafruit.com/about}{Adafruit } ontwikkelt diverse componenten voor met name embedded systemen
die door ieder gebruiker kan worden toegepast en/of geprogrammeerd. Om dit waar te kunnen maken worden veel tutorials ontwikkeld. Zo is er ook een uitgebreide website, om de \href{https://learn.adafruit.com/use-micro-bit-with-arduino}{microbit met de Arduino IDE} te programmeren.

Doordat op de microbit een NRF52 microcontroller zit, zal de betreffende package bij de arduino omgeving bekent moeten worden gemaakt, waarna de betreffende board met deze microcontroller geselecteerd kunnen worden.
\begin{enumerate}
	\item  Ga naar File $\rightarrow$ Preferences, en zet bij Additional boards manager URLs: de URL  %\texttt{https://sandeepmistry.github.io/arduino-nRF5/package\_nRF5\_boards\_index.json
    {\scriptsize \texttt{https://sandeepmistry.github.io/arduino-nRF5/package\textunderscore nRF5\textunderscore boards\textunderscore index.json}}  zoals in figuur \ref{fig:arduinoPref2} te zien is.\\ 

	\begin{figure}[H]
		\captionsetup{justification=centering}
		\includegraphics[width=0.60 \linewidth]{figuren/arduinoPref2}
		\centering
		\caption{preferences scherm.}
		\label{fig:arduinoPref2}
	\end{figure}
	
   \item Het selecteren van diverse borden met een nRF5 microcontroller  wordt als volgt gedaan: Ga naar Tools $\rightarrow$ Boards $\rightarrow$ Boards manager.. zoals figuur \ref{fig:ardTool1} aangeeft.
\begin{figure}[h!]
	\centering
	\begin{center} 	
		\begin{subfigure}[b]{0.48\textwidth}
			\includegraphics[width=0.8\textwidth]{figuren/arduinoTools}
			\caption{selecteren van de board manager }
			\label{fig:ardTool1}
			
		\end{subfigure}
		\begin{subfigure}[b]{0.48\textwidth}
			\includegraphics[width=0.8\textwidth]{figuren/arduinoTools2}
			\caption{Installeren van de NRF5 borden }
			\label{fig:ardTool2}
		\end{subfigure}
		\captionsetup{justification=centering}
		\caption{Arduino laten werken met de microbit. }
		\label{fig:ardTool}
	\end{center}
	
\end{figure}
Installeer vervolgens de \textbf{Nordic Semicinductor nRF}, zoals in figuur \ref{fig:ardTool2} wordt weergeven.

\item Het selecteren van de microbit als het bordje waarop geprogrammeerd wordt, wordt als volgt gedaan: Ga naar Tools $\rightarrow$ Boards $\rightarrow$ Nordic Semiconductor NRF5 Boards en selecteer BBC micro:bit V2 zoals figuur \ref{fig:ardTool3} wordt weergegeven.
\begin{figure}[H]
	\captionsetup{justification=centering}
	\includegraphics[width=0.60 \linewidth]{figuren/arduinoTools3}
	\centering
	\caption{Selecteren van de micro:bit als bord waarop gewerkt wordt.}
	\label{fig:ardTool3}
\end{figure}

\item Controleer of de arduino aangesloten is op de USB port en selecteer de juiste COM port. Ga naar Tools $\rightarrow$ Port $\rightarrow$ (BBC micro:bit,....))

In figuur \ref{fig:ardTool4} is te zien dat de microbit op COM7 zit, dit kan bij iedereen weer anders zijn.
\begin{figure}[H]
	\captionsetup{justification=centering}
	\includegraphics[width=0.60 \linewidth]{figuren/arduinoTools4}
	\centering
	\caption{Selecteren van de micro:bit als bord waarop gewerkt wordt.}
	\label{fig:ardTool4}
\end{figure}

\item Het uploaden van een programma gebeurt door op de knop \img{figuren/ardIcUpl} te klikken. Windows komt met de melding zoals hieronder wordt weergegeven. 
   	\begin{figure}[h!]
   	\captionsetup{justification=centering}
   	\includegraphics[width=0.45 \linewidth]{figuren/windowsDefSec}
   	\centering
   	\caption{Melding van windows defender.}
   	\label{fig:windowsDef}
   \end{figure}
Hiermee vraagt Windows toestemming om de USB port te mogen gebruiken.
\item Haal de voorbeeldcode van blackboard of \href{https://github.com/JohnVi-hhs/embsysP/tree/main/voorbeelden/blink.ino}{download} de laatste versie en plaats deze in je eigen werkdirectory.
\item \label{en:blink} Open het voorbeeldprogramma blink.ino (dubbel klik), dat wordt weergegeven in Listing \ref{lst:blink}, compileer en upload naar de micro:bit (klik op de knop \img{figuren/ardIcUpl}). \\
Als het goed is gaat het linkerboven LEDje van de matrix knipperen. Voor verder uitleg zie hoofdsuk \ref{sec:blink}


\end{enumerate}

\subsection{Het installeren van de Adafruit library.}

Zodra je complexere zaken wilt aanpakken, heb je een hulplibrary nodig om dingen zoals de interne temperatuursensor, LED-matrix of Bluetooth-verbinding te beheren.
Om het je makkelijker te maken, heeft Adafruit een wrapper-library geschreven die dit allemaal voor je regelt.

\begin{enumerate}
	\item Ga naar Sketch$\rightarrow$ include library $\rightarrow$ Manage Labraries zoals figuur \ref{fig:ardLib} aangeeft
	
	   	\begin{figure}[h!]
		\captionsetup{justification=centering}
		\includegraphics[width=0.45 \linewidth]{figuren/arduino_lib}
		\centering
		\caption{Installeren van van Library.}
		\label{fig:ardLib}
	\end{figure}
	\item Zoek naar de Blep library en installeer deze zoals te zien is in figuur \ref{fig:ardlibBl} de Adafruit library + dependency zoals te zien is in figuur \ref{fig:ardlibAdaf} en de Adafruit Microbit zoals te zien is in figuur \ref{fig:ardlibMicro}.
	
\begin{figure}[h!]
	\centering
		\begin{center} 	
			\begin{subfigure}[b]{0.31\textwidth}
				\includegraphics[width=0.98\textwidth]{figuren/arduinoManLibBlep}
				\caption{Installeren van BLEPerpheral }.
				\label{fig:ardlibBl}
				
			\end{subfigure}
			\begin{subfigure}[b]{0.33\textwidth}
				\includegraphics[width=0.98\textwidth]{figuren/arduinoLibAdafruit}
				\caption{Installeren Adafruit microbit }
				\label{fig:ardlibAdaf}
			\end{subfigure}
			\begin{subfigure}[b]{0.33\textwidth}
	\includegraphics[width=0.98\textwidth]{figuren/arduinoLibAdamico}
	\caption{Installeren Adafruit library }
	\label{fig:ardlibMicro}
\end{subfigure}			
			\captionsetup{justification=centering}
			\caption{Het installeren van Arduino libraries. }
			\label{fig:ardInstal}
		\end{center}
		
	\end{figure}
	
	
\end{enumerate}
